\question{Write the equations of the tangent lines from $\left(3,0\right)$ to

\begin{equation}
	x^2+2y^2=6.\label{63a}
\end{equation}}{calculus/63-line-tangent-elipse-given-one-point.html}{13 April 2005}

\paragraph{Solution} \solutionAuthor{ticbol}

The slope of the tangent line is $\frac{dy}{dx}$. Thus, to find $\frac{dy}{dx}$, we differentiate $x^2+2y^2=6$ implicitly with respect to $x$.

\begin{eqnarray*}
	\frac{d}{dx}\left(x^2+2y^2\right)&=&\frac{d}{dx}\left(6\right)\\
	\Rightarrow dx+4y\frac{dy}{dx}&=&0\\
	\Rightarrow 4y\frac{dy}{dx}&=&-2x\\
	\Rightarrow \frac{dy}{dx}&=&-\frac{2x}{4y}\\
	&=&-\frac{x}{2y}
\end{eqnarray*}

Let $\left(x,y\right)$ be the point at which the tangent line touches the ellipse.

We have two points on the tangent line, $\left(3,0\right)$ and $\left(x,y\right)$.

We can determine the slope $m$ of the tangent line from these two points:

\begin{eqnarray*}
	m&=&\frac{y-0}{x-3}\\
	&=&\frac{y}{x-3},
\end{eqnarray*}

which is the same as $\frac{dy}{dx}$.

\begin{eqnarray}
	\therefore-\frac{x}{2y}&=&\frac{y}{x-3}\nonumber\\
	\Rightarrow-x\left(x-3\right)&=&\left(2y\right)y\nonumber\\
	\Rightarrow-x^2+3x=2y^2\label{63b}
\end{eqnarray}

\begin{eqnarray}
	x^2+2y^2&=&6\qquad\mbox{(from \ref{63a})}\nonumber\\
	\Rightarrow2y^2&=&6-x^2\label{63c}
\end{eqnarray}

\begin{eqnarray}
	3x&=&6\qquad\mbox{(substitute \ref{63c} into \ref{63b})}\nonumber\\
	\Rightarrow x&=&\frac{6}{3}\label{63d}\\
	&=&2\nonumber
\end{eqnarray}

\begin{eqnarray*}
	2^2+2y^2&=&6\qquad\mbox{(substitute \ref{63d} into \ref{63a})}\\
	\Rightarrow2y^2&=&6-2^2\\
	&=&2\\
	\Rightarrow y^2&=&\frac{2}{2}\\
	&=&1\\
	\Rightarrow y&=&\pm\sqrt{1}.
\end{eqnarray*}

This means that there are two points at which the tangent line could touch the ellipse: $\left(2,1\right)$ and $\left(2,-1\right)$.

For $\left(2,1\right)$, the slope of the tangent line from $\left(3,0\right)$ is

\[\frac{dy}{dx}=m=-\frac{x}{2y}=-\frac{2}{2}=-1.\]

Using the point-slope form of the equation of a line, we have

\begin{eqnarray*}
    \left(y-y_1\right)&=&m\left(x-x_1\right)\\
    \Rightarrow\left(y-1\right)=-1\left(x-2\right)\qquad\mbox{(using }\left(2,1\right)\mbox{ as }\left(x_1,y_1\right)\\
    \Rightarrow y-1&=&-x+2\\
    \Rightarrow x+y-1-2&=&0\\
    \Rightarrow x+y-3&=&0\qquad\mbox{(the tangent line).}
\end{eqnarray*}

For $\left(2,-1\right)$,

\[\frac{dy}{dx}=m=-\frac{x}{2y}=-\frac{2}{2\times-1}=1.\]

\begin{eqnarray*}
    \left(y-\left(-1\right)\right)&=&1\left(x-2\right)\qquad\mbox{(using the point-slope form with }\left(2,-1\right)\mbox{ as }\left(x_1,y_1\right)\\
    \Rightarrow y+1&=&x-2\\
    \Rightarrow y-x+1+2&=&0\\
    x-y-3&=&0\qquad\mbox{(the tangent line).}
\end{eqnarray*}

Thus the two tangent lines from point $\left(3,0\right)$ to the ellipse given by $x^2+2y^2=6$ are

\[x+y-3=0\]

and

\[x-y-3=0.\]

See figure \vref{fig.63a}.

\begin{figure}\caption{}\begin{center}\label{fig.63a}\includegraphics[scale=0.5]{asy/63.fig1.asy.eps}\end{center}\end{figure}
