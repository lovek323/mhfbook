\url{http://www.mathhelpforum.com/math-help/calculus/36-cannonbal-question.html}

Posted 8 April 2005

\paragraph{Question} A cannonball is fired on a forty-five degree angle from horizontal. The initial position of the cannonball is $\left(0,0\right)$ and its initial velocity is $100\sqrt{2}\mbox{ ft/sec.}$

The cannonball's trajectory is given by

\[y\left(x\right)=x-\left(\frac{x}{25}\right)^2\mbox{ ft.}\]

\begin{enumerate}
	\item
		When will the cannonball hit the ground?
	\item
		What is the maximum height the cannonball will reach?
\end{enumerate}

\paragraph{Solution}

\partSolutionAuthor{paultwang}

\begin{enumerate}
	\item
		To find the $x$ coordinate when the cannonball hits the ground, solve

		\begin{eqnarray*}
			x-\left(\frac{x}{25}\right)^2&=&0\\
			\Rightarrow x\left(1-\frac{x}{125}\right)&=&0\\
			\Rightarrow x\left(125-x\right)&=&0
		\end{eqnarray*}

		\[\therefore x=0\mbox{ or }x=125.\]

		We would expect $y=0$ when $x=0$ as the cannonball starts at the ground. Thus, the cannonball reaches the ground 125 seconds after it was fired.

	\item
		To find the maximum height the cannonball will reach, we use the \emph{first derivative test} to locate a turning point (which will be the maximum).

		\begin{eqnarray*}
			\frac{dy}{dx}&=&\frac{d}{dx}\left(x-\frac{1}{125}x^2\right)\\
			&=&1-\frac{2x}{125}
		\end{eqnarray*}

		Now we find the zero of this equation

		\[y^\prime\left(62.5\right)=0\]

		and plug this back into the original equation

		\begin{eqnarray*}
			y\left(62.5\right)&=&62.5-\left(\frac{62.5}{25}\right)^2\\
			&=&56.25.
		\end{eqnarray*}

		The cannonball will reach a maximum height of 56.25 feet.
\end{enumerate}
